\documentclass[10pt,a4paper]{report}
%\usepackage[utf8]{inputenc}
\usepackage{graphicx}
\usepackage{float}
\usepackage{tikz}
\usetikzlibrary{calc}
\usepackage{hyperref}

\graphicspath{{./images/}}

\begin{document}
\pagenumbering{gobble}% Remove page numbers (and reset to 1)
\clearpage
\thispagestyle{empty}

\begin{tikzpicture}[overlay, remember picture]
\draw[line width =4pt,rounded corners=15pt,double,blue]
($(current page.north west)+ (0.5in,-0.5in)$)
rectangle
($(current page.south east)+(-0.5in,0.5in)$);


\end{tikzpicture}


\begin{center}
\begin{Huge}
\textbf{PIC18F4550}\par
\vspace{0.7cm}
Bootloader Installation
\end{Huge}
\vspace{1.5cm}
\author{Shreyas Deshpande}
\end{center}

\begin{figure}[H]
 \centering
 \includegraphics[scale=0.5]{PIC.png}
\end{figure}
\hspace{3.5cm}
\begin{center}
\begin{Large}
\textbf{By:}\par
\textbf{Shreyas Deshpande}
\end{Large}
\end{center}

\newpage

\pagenumbering{Roman}

\begin{normalsize}

\tableofcontents

\end{normalsize}

\newpage

\thispagestyle{plain}

\pagenumbering{arabic}

\chapter{Getting Started...}
\section{Pre-requisites}
Before we start, its always better to collect the tools and other necessory stuff required to achive end result.

\begin{itemize}
\item PIC18F4550 Developement Board
\item PICKIT-3
\item PICKIT-3 PC Application
\item Bootloader Hex file
\item Bootloader device driver
\item PDFSUSB Application \texttt{[For hex file upload]}
\item LED blink demo hex file \texttt{[Optional]}
\end{itemize}

\section{Brief Procedure}
We have provided you the folder with alll the \textit{installation} files, \textit{Hex} files and \textit{drivers}. Please Navigate to that folder and then follow the procedure given bellow.

\begin{enumerate}
\item \label{Step_1}Install the \textit{PICKIT-3 PC Application}
\item Connect the \textit{PICKIT-3} with \textit{PIC18F4550} micro-controller. 
\item Open the \textit{PICKIT-3 PC Application} installed in Step \ref{Step_1}.
\item Upload the \textit{Bootloader hex file} into PIC18F4550 micro-controller.
\item Remove the \textit{PICKIT-3} Programmer from micro-controller.
\item Install the \textit{Bootloader Device Drivers} on the Windows.
\item Put the PIC18F4550 Microcontroller in the program mode.
\item Open the PDFSUSB Application
\item Select the microcntroller
\item Select the LED blink demo hex file or any other hex file you want to upload on the micro-controller.
\item Click on the Program button, and wait for the hex file to get upload in the micro-controller.
\item Reset the micro-controler to observe the output.
\end{enumerate}

This is the brief procedure for power users who are used to the \textit{Microchip's} and \textit{PIC microcontroller's} eco-system.\par
If you are not familier with this eco-system or you find the above procedure hard, don't worry, I have given the detailed and stepwise procedure in later part of this document, which will guide you making your PIC bootloader experience buttery smooth. 


\chapter{Installing PICKIT-3 Application}
This chapter guides you step-by-step to install the \textit{PICKIT-3 Application.} Follow the procedure as mentioned bellow and at the end you will end up with \textit{PICKIT-3 Application} installed in your PC succesfully.\par

\section{Installation Procedure}
\label{install_pickit3_app}
\begin{enumerate}
\item Navigate to the \textit{PICKIT-3 Application's} setup file location and double click on the exe file. After opening the exe file, a new window will appear as shown in figure \ref{pickit3_installer_1}.

\begin{figure}[H]
\centering
\includegraphics[scale=0.45]{Pickit3_app1.jpeg}
\caption{Welcome Page}
\label{pickit3_installer_1}
\end{figure}

\item On this window, click on the \textit{Next} button.

\begin{figure}[H]
\centering
\includegraphics[scale=0.45]{Pickit3_app2.jpeg}
\caption{Select Installation Location}
\label{pickit3_installer_2}
\end{figure}


\item As we don't need to change the default location, just click on the \textit{Next} button, without changing any option shown in figure \ref{pickit3_installer_2}.

\begin{figure}[H]
\centering
\includegraphics[scale=0.45]{Pickit3_app3.jpeg}
\caption{Confirm Installtion}
\label{pickit3_installer_3}
\end{figure}

\item Here, the installer is all set to install the application on the PC, click on the \textit{Next} button.

\begin{figure}[H]
\centering
\includegraphics[scale=1.8]{Pickit3_app4.jpeg}
\caption{License Agreement}
\label{pickit3_installer_4}
\end{figure}

\item Next window appeared is of \textit{License Agreement}, Select the \textit{I Agree} radio button and then click on \textit{Next}. \texttt{[Refer Figure \ref{pickit3_installer_4}.]}

\begin{figure}[H]
\centering
\includegraphics[scale=1.8]{Pickit3_app5.jpeg}
\caption{Installation Complete}
\label{pickit3_installer_5}
\end{figure}

\item We are now at last stage of our installation. At the end of the installation, the installer present you the window as shown in the figure \ref{pickit3_installer_5}. Click on the \textit{Close} button to close the installer.


\end{enumerate}

Congratulations !!! ... You have successfully installed the \textit{PICKIT-3 Application} on your PC.

\newpage
\section{Uploading Bootloader}
\label{sec:Upload_Bootloader}
Now you have installed the apllication, it's time to use that application to upload the PIC USB bootloder in to the \textit{PIC18F4550} micro-controller.\par
Follow the steps given bellow to upload the USB bootloader.

\begin{enumerate}

\item If PIC18F4550 micro-controller is powerd ON, turn it OFF.
\item Connect the PICKIT-3 to micro-controller.
\item Now turn ON the micro-controller.
\item Connect the \textit{PICKIT-3} to PC using USB Cable.

\item Open the \textit{PICKIT-3 Application} we have installed in section \ref{install_pickit3_app}. A new Window will appear as shown in Figure \ref{Upload_Bootloader_1}.

\begin{figure}[H]
\centering
\includegraphics[scale=1.8]{Pickit3_app6.jpeg}
\caption{PICkit-3 Application}
\label{Upload_Bootloader_1}
\end{figure}

\item Now Navigate to \texttt{Tools -> Check Communication} as shown in the Figure \ref{Upload_Bootloader_2}.

\begin{figure}[H]
\centering
\includegraphics[scale=1.8]{Pickit3_app7.jpeg}
\caption{Check Connected Devices}
\label{Upload_Bootloader_2}
\end{figure}



\item The application will detect the PICKIT-3 and the micro-controller connected through it as shhown in the Figure \ref{Upload_Bootloader_3}.

\begin{figure}[H]
\centering
\includegraphics[scale=1.8]{Pickit3_app8.jpeg}
\caption{Device Detected}
\label{Upload_Bootloader_3}
\end{figure}

\item Now We need to select the hex file of the bootloader. To do so, navigate to \texttt{File -> Import File} \textsl{Add image here(Pending)}

\item Now navigate to the location where you have stored the bootloader's hex file and select the hex file as shown in the Figure \ref{Upload_Bootloader_5}. Here the filename of the hex file is \texttt{PIC-USB-4550-BOOT.hex}

\begin{figure}[H]
\centering
\includegraphics[scale=1.8]{Pickit3_app9.jpeg}
\caption{Select bootloader hex file}
\label{Upload_Bootloader_5}
\end{figure}

\item Now Click on the \textit{Write} button as shown in the figure \ref{Upload_Bootloader_6}.

\begin{figure}[H]
\centering
\includegraphics[scale=1.8]{Pickit3_app11.jpeg}
\caption{Write Hex File}
\label{Upload_Bootloader_6}
\end{figure}


\item Once hex file gets written on the micreocontroller successfully, the success message gets displayed on the application's message window as shown in figure \ref{Upload_Bootloader_7}.

\begin{figure}[H]
\centering
\includegraphics[scale=1.8]{Pickit3_app12.jpeg}
\caption{Hex File Uploaded Succesfully}
\label{Upload_Bootloader_7}
\end{figure}

\item Done !!!. The bootloader is succesfully get uploaded into the microcontroller. Now close the Application

\item Remove the \textit{PICKIT-3} from PC as well as from micro-controller.


\end{enumerate}

\chapter{Installing MCHPFSUSB v2.2 USB Framework}
\label{chap:MCHPFSUSB}
In this chapter we are going to install the MCHPFSUSB USB Framework of version 2.2. This Framework provides USB bootloader drivers as well as firmware hex file upload tool.\par

Following steps guide you to install the MCHPFSUSB USB Framework on your PC.

\begin{enumerate}
\item Navigate to the MCHPFSUSB installer folder and run the setup file as shown in the Figure .

\begin{figure}[H]
\centering
\includegraphics[scale=1.5]{MCHPFSUSB1.jpeg}
\caption{Select Setup File}
\label{MCHPFSUSB_1}
\end{figure}

\item Installer will invoke and ask you for the License Agreement as shown in the Fuigure \ref{MCHPFSUSB_2}. Click on \textit{'I Accept'}. 

\begin{figure}[H]
\centering
\includegraphics[scale=1.5]{MCHPFSUSB2.jpeg}
\caption{License Agreement}
\label{MCHPFSUSB_2}
\end{figure}

\texttt{[Note]: There are two different license agreement and simmilar window appears for tow times and every time you have to Accept the Agreement}

\item Then the installer present you the welcome window as shown in Figure \ref{MCHPFSUSB_3}. Which warns you close any other open window before proceeding in installation. So close all other windows as instructed in the installer and then click on \textit{Next} button.

\begin{figure}[H]
\centering
\includegraphics[scale=1.5]{MCHPFSUSB3.jpeg}
\caption{Welcome Window}
\label{MCHPFSUSB_3}
\end{figure}

\item In the next window of the installer you have to choose the destinaton location. Don't change anything and click on the \textit{Next} button. \texttt{[Ref. Figure \ref{MCHPFSUSB_4}}.

\begin{figure}[H]
\centering
\includegraphics[scale=1.5]{MCHPFSUSB4.jpeg}
\caption{Choose Installtion Location}
\label{MCHPFSUSB_4}
\end{figure}

\item It's good practice to keep the backup of the files which are already present on the computer and will get replace in further installation process. The setup is asking whether to take backup or not. By defualt \textit{Yes} option is selected, which is a good choice so we will not change anything on this page and click on \textit{Next} to move ahead. \texttt{[Ref. Figure \ref{MCHPFSUSB_5}]}

\begin{figure}[H]
\centering
\includegraphics[scale=1.5]{MCHPFSUSB5.jpeg}
\caption{Take a Backup}
\label{MCHPFSUSB_5}
\end{figure}

\item At this point installer is ready to install the \textbf{MCHPFSUSB Framework} on your PC. In this window it will take confirmation from you as shown in the Figure \ref{MCHPFSUSB_6}. Click on \textit{Next} button.

\begin{figure}[H]
\centering
\includegraphics[scale=1.5]{MCHPFSUSB6.jpeg}
\caption{Setup Confirmation}
\label{MCHPFSUSB_6}
\end{figure}

\item The installation is now started and will install the all necessory files and tools on your PC. \texttt{[Ref. Figure \ref{MCHPFSUSB_7}]}

\begin{figure}[H]
\centering
\includegraphics[scale=1.5]{MCHPFSUSB7.jpeg}
\caption{Installation in Progress}
\label{MCHPFSUSB_7}
\end{figure}\

\item After installation process gets completed, the installer will present you the finishing window as shown in the Figure \ref{MCHPFSUSB_8}. 

\begin{figure}[H]
\centering
\includegraphics[scale=1.4]{MCHPFSUSB8.jpeg}
\caption{Installation Complete}
\label{MCHPFSUSB_8}
\end{figure}

Click on the \textit{Finish} button to close the installer.

\end{enumerate}

\chapter{Installing Driver}
In last \autoref{chap:MCHPFSUSB} we have completed the \textbf{MCHPFSUSB Framework}. Now it's time to install the drivers required to use \textit{PIC USB Bootloader}.\par
For this you will required to connect the PIC18F4550 micro-controller to PC on which we have uploaded the bootloader's hex file, previously. \texttt{[Refer \autoref{sec:Upload_Bootloader}]}

\begin{enumerate}
\item Connect the PIC18F4550 microcontroller to the PC using USB cable. Also connect the Power supply to controller board and turn the board ON.
\item Now we need to put the controller into bootloader mode / Programming mode.
\item To put the board into the Programming moed, press and hold the User button connected to \textbf{PB4} pin, and then press and release the \textbf{Reset} button. then release the \textbf{User button}.

\item Now right click on \textit{This PC}, click on \textit{Manage}, then select \textit{Device Manager}. 

\begin{figure}[H]
\centering
\includegraphics[scale=1.6]{Device_Driver1.jpeg}
\caption{Device Manager}
\label{Driver_1}
\end{figure}

\item The device manager will show unknown device as shown in figure \ref{Driver_1}. as we have not installed the driver yet.

\begin{figure}[H]
\centering
\includegraphics[scale=1.6]{Device_Driver2.jpeg}
\caption{Update Driver}
\label{Driver_2}
\end{figure}

\item Right click on the \textit{Unknown Device}  and select \textit{Update Driver} as shown in figure \ref{Driver_2}.

\begin{figure}[H]
\centering
\includegraphics[scale=1.6]{Device_Driver3.jpeg}
\caption{Browse Folder}
\label{Driver_3}
\end{figure}

\item On next window select \textit{Browse my conputer for drivers}, as shown in figure \ref{Driver_3}.

\begin{figure}[H]
\centering
\includegraphics[scale=1.6]{Device_Driver4.jpeg}
\caption{Driver Location}
\label{Driver_4}
\end{figure}

\item Now \textit{Device Manager} asks for the driver location. to select the driver folder, click on \textit{Browse} button as shown in the figure \ref{Driver_4}.

\begin{figure}[H]
\centering
\includegraphics[scale=1.6]{Device_Driver5.jpeg}
\caption{Select Folder}
\label{Driver_5}
\end{figure}

\item After clicking on \textit{Browse} button, navigate to \texttt{C:\textbackslash Microchip Solutions} as shown in figure \ref{Driver_5}.

\begin{figure}[H]
\centering
\includegraphics[scale=1.6]{Device_Driver6.jpeg}
\caption{Include Subfolders}
\label{Driver_6}
\end{figure}

\item Make sure that \textit{Include Subfolders} option is selected as shown in the figure \ref{Driver_6} and then click on \textit{Next} button.

\begin{figure}[H]
\centering
\includegraphics[scale=1.6]{Device_Driver7.jpeg}
\caption{Driver Installed}
\label{Driver_7}
\end{figure}

\item Now wait for some time to driver gets installed. Once the drivers get installed on the PC, the device manager will show you the window as shown in figure \ref{Driver_7}. Click on the \textit{Close} button to close the driver installer window.

\begin{figure}[H]
\centering
\includegraphics[scale=1.6]{Device_Driver8.jpeg}
\caption{Confirm the installed driver}
\label{Driver_8}
\end{figure}

\item you can now see the controller is detected by \textit{Device Manager} as \texttt{Microchip Custom USB Device} as shown in figure \ref{Driver_8}. Which confirms that driver installation is succesfull.

\end{enumerate}
 
\chapter{Uploading firmware hex file}

\appendix
\chapter{Schematic}

\end{document}